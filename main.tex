\documentclass[10pt]{article}
% packages
\usepackage[left=1.87cm,right=1.87cm,top=1.87cm,bottom=1.87cm]{geometry}
\usepackage[utf8]{inputenc}
\usepackage{mathrsfs}
\usepackage{bm}
\usepackage{multirow}
\usepackage{amsfonts}
\usepackage{amssymb}
\usepackage{amsmath}
\usepackage{float}
\usepackage{mathtools}
\usepackage{amsthm} 
\usepackage{comment}
\usepackage{xcolor}
\usepackage{graphicx}
\usepackage{yhmath}
\usepackage{mathdots}
\usepackage{wasysym}
\usepackage{xargs}
\usepackage[symbol]{footmisc}
\usepackage[shadow,textsize=tiny,tickmarkheight=4pt]{todonotes}

% HURL
\usepackage{hyperref}
\hypersetup{
    colorlinks=true,
    linkcolor=red,
    filecolor=red,      
    urlcolor=red,
    citecolor=red
}
\urlstyle{same}

% Theorem environments
\theoremstyle{definition}
\newtheorem{theorem}{Theorem}
\newtheorem{claim}{Claim}
\newtheorem{proposition}{Proposition}
\newtheorem{lemma}{Lemma}
\newtheorem{remark}{Remark}
\newtheorem{definition}{Definition}

% Math commands
\newcommand{\F}{\mathbb{F}}
\newcommand{\N}{\mathbb{N}}
\newcommand{\C}{\mathbb{C}}
\newcommand{\R}{\mathbb{R}}
\renewcommand{\P}{\mathbb{P}}
\newcommand{\Fp}{\mathbb{F}_p}
\newcommand{\Fpp}{\mathbb{F}_{p^2}}
\newcommand{\vp}{\varphi}
\newcommand{\ra}{\rightarrow}
\newcommand{\vph}{\hat{\varphi}}
\newcommand{\Z}{\mathbb{Z}}

% Keywords command
\providecommand{\keywords}[1]
{
  \smallskip\noindent\small	
  \textbf{\textbf{Keywords---}} #1
}
\newcommand{\LC}[2]{\ensuremath{L_{[#1 : #2]}}}
\newcommand{\Mont}[2]{\ensuremath{M_{#1, #2}}}
\newcommand{\Id}{\mathcal{O}}

% other shortcuts 
\newcommand\mytodo[1]{(\textcolor{red}{{\bf todo}}: #1)}
\def\td{(\textcolor{red}{{\bf todo}})}
\newcommand\done[1]{- \textcolor{red}{{\bf done}} #1}
\def\done{(\textcolor{red}{{\bf done}})}
\def\td{(\textcolor{red}{{\bf todo}}) }
\def\why{(\textcolor{red}{{\bf why?}}) }
\renewcommand{\thefootnote}{\fnsymbol{footnote}}
\DeclarePairedDelimiter{\ceil}{\lceil}{\rceil}
\DeclarePairedDelimiter{\floor}{\lfloor}{\rfloor}
\DeclareMathOperator{\cswap}{cswap}
\DeclareMathOperator{\End}{End}
\DeclareMathOperator{\chr}{char}
\DeclareMathOperator{\bd}{bd}
\DeclareMathOperator{\crit}{crit}


% Algorithms
\usepackage[vlined,ruled,linesnumbered,titlenotnumbered]{algorithm2e}
%\usepackage{algorithm2e}
\newcommand{\Input}{\item[\textsc{Input:}]}
\newcommand{\Output}{\item[\textsc{Output:}]}
\newcommand{\Remark}{\item[\textsc{Remark:}]}
\newcommand{\pushline}{\Indp}% Indent
\newcommand{\popline}{\Indm\dosemic}% Undent
\let\oldnl\nl% Store \nl in \oldnl
\newcommand{\nonl}{\renewcommand{\nl}{\let\nl\oldnl}}% Remove line number for one line
\makeatother


% Tikz
\usepackage{tikz}
\usetikzlibrary{arrows.meta}




\title{A las vegas algorithm for computing one point in each connected component of a smooth real algebraic set}
\author{Jesse Elliott\thanks{David R. Cheriton School of Computer Science, University of Waterloo, On, Canada}, Mohab Safey El Din\thanks{Sorbonne Universit\'e, CNRS, INRIA, Laboratoire d’Informatique de Paris 6, PolSys, Paris, France}, \'Eric Schost \thanks{David R. Cheriton School of Computer Science, University of Waterloo, On, Canada} 
}


\date{}

\def\ms#1{\textcolor{olive}{#1}}
\begin{document}

\maketitle




\begin{abstract}
This is an abstract \newline
\keywords{Real algebraic geometry, polynomial system solving, algorithms}
\end{abstract}



%%%%%%%%%%%%%%%%%%%%%%%%%%%%
\section{Preliminaries and notation}
For $i$ in $\{1,\dots,n\}$ and $\{i_1,\hdots,i_d\} \subset \{1,\dots,n\},$ consider the projections
%
\begin{align*}
    \vp_i: \C^n  &\rightarrow \C^i \\
    (x_1,\hdots,x_n) &\mapsto  (x_1,\hdots,x_i)    
\end{align*}
%
%
\begin{align*}
    \psi_i: \C^n  &\rightarrow \C^{n-1} \\
    (x_1,\hdots,x_n) &\mapsto  (x_1,\hdots,x_{i-1},x_{i+1},\hdots,x_n),    
\end{align*}
%
\begin{align*}
    \pi_{i_1,\hdots,i_d}: \C^n  &\rightarrow \C^d \\
    (x_1,\hdots,x_n) &\mapsto  (x_{i_1},\hdots,x_{i_d}).    
\end{align*}
%
For a variety $V \subset \C^n$, consider critical points $\crit(\pi_{i_1,\hdots,i_d},V) = \big\{\bm x \in V~|~\dim \left(\pi_{i_1,\hdots,i_d}\left(T_{\bm x} \right) \right) < d \big\}$ where $T_{\bm x}(V)$ is the Zariski tangent space at $\bm x \in V$.
%
%
\begin{definition}
Let $S \subset \R^n$ be a real algebraic set. The \textit{dimension} of $S$ is the largest integer $d$ with the property that, for any $\{i_1,\hdots,i_d\} \subset \{1,\dots,n\}, \pi_{i_1,\hdots,i_d}(S)$ contains an open subset of $\R^d$.     
\end{definition}
%





%%%%%%%%%%%%%%%%%%%%%%%%%%%%
\section{Removing the Noether position requirement}
In this section we prove the following proposition. 
%
\begin{proposition}\label{prop:main}
Let $V \subset \C^n$ be a smooth variety and let $C$ be a connected component of $V \cap \R^n$ with dimension $d$. If, for all $\{i_1,\hdots,i_d\} \subset \{1,\dots,n\}$, 
\begin{enumerate}
    \item $C \cap \crit(\pi_{i_1,\hdots,i_d},V) = \emptyset$;
    \item $\pi_{i_1,\hdots,i_d}(C) \not = \R^d;$
\end{enumerate}
then there exists distinct $i,j \in\{1,\hdots,n\}$ with $C \cap \{ x_i= x_j\} \not = \emptyset$ or 
$C \cap \{ x_i= -x_j\} \not = \emptyset.$
\end{proposition}
%





%%%%%%%%%%%%%%%%%%%%%%%%%%
\subsection{Dimension $1$ case}
For $d=1$, we assume that for all $i \in \{1,\hdots,n\},$ 
%
\begin{enumerate}
    \item $C \cap \crit(\pi_i,V) = \emptyset$;
    \item $\pi_i(C) \not = \R$.
\end{enumerate}
%
\noindent
Notice that, for all $i \in \{1,\hdots,n\},$ since $\crit(\pi_i,C) = \emptyset$ and $\pi_i(C) \not = \R$, $\pi_i(C)$ is an open interval:
\[
\pi_i(C)=
\begin{cases}
			(e_i,+\infty), \\
                (-\infty,e_i),\textrm{ or }\\
                (e_i,e_i').
\end{cases}
\]
%
\begin{lemma}\label{lemma:unboundedCoordinate}
    There exists $i \in \{1,\hdots,n\}$ with $\pi_i(C)$ not bounded. (WLOG assume $i=1.$)
\end{lemma}
%
%
\begin{proof}
Assume for contradiction that $\pi_i(C)$ is bounded for all $i \in \{1,\hdots,n\}$, and let $\pi_i(C) = (e_i,e_i').$
We know that $C$ is therefore closed because the connected components of a curve are also closed, and bounded, and therefore compact. It follows that $\pi_i(C)$ are all compact (because $\pi_i$ are continuous), which contradicts that $\pi_i(C) = (e_i,e_i')$ are all open. 
\end{proof}
\noindent 
%
By the semi-algebraic implicit function theorem, we can obtain a parameterization
\[
\big\{\vartheta(t) = \left(t,g_2(t),\hdots,g_n(t)\right)~|~ t \in (e_1,+\infty)\big\} \subset C.
\]
where $g_2(t),\hdots,g_n(t)$ are semi-algebraic functions. Furthermore, since $C$ is smooth, $g_2(t),\hdots,g_n(t)$ are diffeomorphisms.
%
\begin{lemma}
The functions $g_2(t),\hdots,g_n(t)$ are monotonic. 
\end{lemma}
%
%
\begin{proof}
Consider the tangent vector to $\vartheta(t)$,
at $C$, at a point in $\crit(\pi_i,V)$; it would give $
\begin{pmatrix}
    1, 
    g'_2(t), 
    \hdots, 
    g'_n(t)
\end{pmatrix},
$ with
%\todo{Not so clear; I believe there is a mistake. Only one partial derivative cancels.} 
$g'_i(t)=0.$ Therefore, since by assumption $C \cap \crit(\pi_i,V) = \emptyset$ for all $i \in \{1,\hdots,n\}$,
$g_2',\hdots,g_n'$ never vanish. It follows that $g_2,\hdots,g_n$ are  monotonic.
\end{proof}
%
\noindent
Now, consider the set $N_1 = \big\{i~|~g_i(t) \rightarrow \infty \textrm{ as } t \rightarrow e_1, \textrm{ and } g_i(t) \not \rightarrow \infty \textrm{ as } t \rightarrow \infty\big\}.$
%
\begin{lemma}
    The set $N_1$ is not empty. 
\end{lemma}
%
\begin{proof}
It follows from Lemma \ref{lemma:unboundedCoordinate} that there exists an index $i$ in $\{2,\hdots,n\}$ with $g_i(t) \rightarrow \infty$ as $t
\rightarrow e_1$.
%\ms{Yes. The idea is to say that, if for some $i$ $\pi_i(\vartheta(t))$ tends to $\infty$ when $t\to e_1$, then $i\in N_1$ because $g_i$ is monotone (and $\pi_i(C)\neq \R$)}. 
WLOG assume $i=2.$ Since
$g_2(t)$ is monotonic, as $t$ ranges from $e_1$ to $+\infty, g_2(t)$ is decreasing and thus monotonically decreasing. We cannot have that $g_2(t)$ tends to $-\infty$ because then as $t$ tends from $e_1$ to $+\infty$, $g_2(t)$ would cover the entire real line, violating our assumption that $\pi_2(C) \not = \R$. Therefore, $g_2(t) \rightarrow e_2$ as $t$ tends from $e_1$ to $+\infty$.
\end{proof}
%







%%%%%%%%%%%%%%%
\paragraph*{Conclusion.}
Since we have shown that $g_2(t) \rightarrow \infty$ as $t \rightarrow e_1$ and $g_2(t) \rightarrow e_2$ as $t \rightarrow \infty$, $g_2(t)$ must cross the hyperplane $x_1-x_2=0$.  Therefore $C$ meets $x_1-x_2=0$, proving Proposition \ref{prop:main} for the $d=1$ case.  





%%%%%%%%%%%%%%%%%%%%%%%%%%%%
\subsection{Dimension $d>1$ case}
Let $\{i_1,\hdots,i_d\} \subset \{1,\hdots,n\},$ and let $\pi$ denote $\pi_{i_1,\hdots,i_d}.$ We can parameterize the connected component of dimension $d$ with $d$ coordinates assuming implicitly that the projection of this connected component on this $d$-dimensional space has non-empty interior. We will always find a set of coordinates to satisfy this. \textbf{If we assume $C \cap \crit(\pi_i,V) = \emptyset$ then it is clear.} By the semi-algebraic implicit function theorem, we can now obtain a parameterization
\[
\big\{\vartheta(t_1,\hdots,t_d) = \left(t_1,\hdots,t_d,g_{d+1}(t_1,\hdots,t_d),\hdots,g_{n}(t_1,\hdots,t_d)\right)~|~ (t_1,\hdots,t_d) \in \pi(C)\big\} \subset \C^n.
\]
where $g_{d+1}(t),\hdots,g_n(t)$ are semi-algebraic functions.  Furthermore, $g_{d+1}(t),\hdots,g_n(t)$ are diffeomorphisms because $C$ is smooth. Now, since we assume that $\pi(C) \not = \R^d,$ we know that $\pi(C)$ has boundary points, which we select as follows. Put $S = \bd\left( \pi(C) \right)$ and note that $\dim S \leq \dim \overline{S}.$ Take a real, regular point $\alpha \in \overline{S}.$ Locally, the boundary at $\alpha$ will look like an affine linear space (with a tangent space). Now, by this reasoning together with the \textit{Curve Selection Lemma} \cite{CurveSelectionLemma},  
we obtain a line segment $\gamma: (0,1] \rightarrow \pi(C)$ that can be extended continuously to $\gamma(0)=\alpha.$ 
%
\begin{lemma}
For $i \in \{1,\hdots,n\}$, the functions $g_i \circ \gamma$ have no critical points. 
\end{lemma}
%
%
\begin{proof}
\td We argue that if $g_i \circ \gamma$ has a critical point then $C \cap \crit(\pi_i,V) \not = \emptyset$ which is a contradiction. 
\begin{enumerate}
    \item All $g_i$ belong to $C$
    \item $g_i \circ \gamma$ are smooth which implies that $\frac{d}{dt}\left(g_i \circ \gamma \right)$ exists everywhere. 
    \item Critical points of $g_i \circ \gamma$ are the same as critical points in $C \cap \crit(\pi_i,V) \not = \emptyset$.
    \item If we assume that $g_i \circ \gamma$ has a critical point then 
\end{enumerate}
\end{proof}
%
%
\begin{lemma}
For $i \in \{1,\hdots,n\}$, the functions $g_i \circ \gamma$ are monotonic. 
\end{lemma}
%
%
\begin{proof}
Since $g_i \circ \gamma$ has no critical points, for all $t \in [0,1], \frac{d}{dt}\left(g_i \circ \gamma\right)(t)\not =0.$ It follows that $g_i \circ \gamma$ are monotonic. 
\end{proof}
%
\noindent 
Put $Z = C \cap \pi^{-1}(\gamma).$ 
%
\begin{lemma}
The dimension of $Z$ is $1$. 
\end{lemma}
%
%
\begin{proof}
Let $d_Z$ be the dimension of $Z$; it is clear that $1 \leq d_Z \leq d$. By the 
\textit{Theorem of the Dimension of Fibres}\cite{Shafarevich}, for a generic $\bm y \in \C^d$, $C \cap \pi^{-1}(\bm y)$ is finite. Hence, the dimension of $C \cap \pi^{-1}(\gamma)$ is at most $1$ and $d_Z=1$. Now consider that $\bm y \in \C^d$ is not generic \td.


\begin{enumerate}
    \item Dimension could be less than $1$. Sin e there are no critical points, the dimension cannot be $0$. Why not empty (then $d_Z = -1$)?
    \item Cannot use \textit{Theorem of the Dimension of Fibres} because $Z$ is a semi-algebraic set. There is a real variant in \textit{Algorithms for real algebraic geometry} 
\end{enumerate}


\end{proof}
%
%
\begin{lemma}\label{lemma:DimensionOneAssumptionsHold}
For all $i \in \{1,\hdots,n\}$, 
\begin{enumerate}
    \item $Z \cap \crit(\pi_i,V) = \emptyset$;
    \item $\pi_i(Z) \not = \R$.
\end{enumerate}
\end{lemma}
%
%
\begin{proof}
If $Z \cap \crit(\pi_i,V) \not = \emptyset$ then since $Z \subset C$ we contradict our assumption that $C \cap \crit(\pi_i,V) = \emptyset$. Similarily, if $\pi_i(Z) = \R$ then we contradict our assumption that $\pi_i(C) \not = \R.$  
\end{proof}
%









%%%%%%%%%%%%%%%
\paragraph*{Conclusion.}
It follows that distinct $i,j \in\{1,\hdots,n\}$ exist with $Z \cap \{ x_i= x_j\} \not = \emptyset$ or 
$Z \cap \{ x_i= -x_j\} \not = \emptyset.$ Since $Z \subset C,$ we have proven Proposition \ref{prop:main}.
















\paragraph*{Notes.}
\ms{We can try to extend the reasoning this way. We consider the projection on
the $(x_1, \ldots, x_d)$-space. Using again something like the semi-algebraic implicit function theorem, we should be able to parameterize all coordinates points in $C$ with $d$-variate continuous semi-algebraic functions $g_{d+1}, \ldots, g_n$ (provided that $C$ does not intersect the polar variety associated to the projection on this $(x_1, \ldots, x_d)$-space). Since $C$ is
  assumed to be smooth, these $g_i$'s should be diffeomorphisms.}

\ms{Now we take $\alpha$ in the boundary of the projection $\pi(C)$ of $C$ on
  the $(x_1, \ldots, x_d)$-space. Assuming additionally that this projection
  does not cover $\R^d$ such a point exists. We can now consider a
  semi-algebraic path $\gamma:(0, 1]\to \pi(C)$ such that $t\mapsto \gamma(t)$
  can be extended continuously to $0$ with $\gamma(0)=\alpha$. }

\ms{My hope is now that $g_i\circ \gamma$ is monotone (or at least that we can
  choose $\alpha$ and $\gamma$ this way ; if $\alpha$ is a ``regular'' point of
  the boundary, we could take $\gamma$ as a segment). It should be ok also to
  show that $\pi^{-1}(\gamma)\cap C$ is $1$-dimensional. }

\ms{What remains then to be done is to show that the fact that $C$ does not meet
any polar variety associated to projections on $d$-dimensional coordinate
subspaces will ``transfer'' to $C\cap \pi^{-1}(\gamma)$ the properties needed to
apply the result above (in the case $d = 1$).}
%













%%%%%%%%%%%%%%%
\section{Practical deterministic algorithms}



%%%%%%%%%%%%%%%
\subsection{Computing the dimension of polar varieties}




%%%%%%%%%%%%%%%
\subsection{Counting real roots of zero-dimensional polynomial systems}
\begin{enumerate}
    \item In \cite{BasuPollackRoy2006} there are deterministic algorithms with complexity $d^{O(n)}$. Our goal is to obtain a practical algorithm with sharper complexity. 
    \item 
\end{enumerate}



%%%%%%%%%%%%%%%
\subsection{Deciding if algebraic sets have finitely many complex solutions}
No algorithms in \cite{BasuPollackRoy2006} for this problem. 



\newpage 
\bibliographystyle{plain}
\bibliography{references.bib}
\newpage 





\end{document}
