\documentclass[11pt]{article}
%\usepackage{fullpage}
\usepackage[left=1.87cm,right=1.87cm,top=1.87cm,bottom=1.87cm]{geometry}

%%%%%%%%%%%%%%%%%%%%%%%%%%%%%%%%%%%%%%%%%%%%%%
\usepackage[utf8]{inputenc}
%\usepackage[left=1in,right=1in,top=1in,bottom=1in]{geometry}

% AMS packages
\usepackage{multirow}
\usepackage{amsfonts}
\usepackage{amssymb}
\usepackage{amsmath}
\usepackage{float}%
\usepackage{mathtools}
\usepackage{amsthm} % Conflicts with llncs
\usepackage{comment}
% Colors
\usepackage{xcolor}
\usepackage{graphicx}
\usepackage{yhmath}
\usepackage{mathdots}
\usepackage{wasysym}
%\usepackage{MnSymbol}
%\usepackage[dvipsnames]{xcolor}

% HURL
\usepackage{hyperref}
\hypersetup{
    colorlinks=true,
    linkcolor=black,
    filecolor=red,      
    urlcolor=red,
    citecolor=blue
}
\urlstyle{same}

% Theorem environments
\theoremstyle{definition}
\newtheorem{theorem}{Theorem}
\newtheorem{claim}{Claim}
\newtheorem{proposition}{Proposition}
\newtheorem{lemma}{Lemma}
\newtheorem{remark}{Remark}
\newtheorem{definition}{Definition}

% Math commands
\newcommand{\F}{\mathbb{F}}
\newcommand{\N}{\mathbb{N}}
\newcommand{\C}{\mathbb{C}}
\newcommand{\R}{\mathbb{R}}
\renewcommand{\P}{\mathbb{P}}
\newcommand{\Fp}{\mathbb{F}_p}
\newcommand{\Fpp}{\mathbb{F}_{p^2}}
\newcommand{\vp}{\varphi}
\newcommand{\ra}{\rightarrow}
\newcommand{\vph}{\hat{\varphi}}
\newcommand{\Z}{\mathbb{Z}}

% Keywords command
\providecommand{\keywords}[1]
{
  \smallskip\noindent\small	
  \textbf{\textbf{Keywords---}} #1
}
\newcommand{\LC}[2]{\ensuremath{L_{[#1 : #2]}}}
\newcommand{\Mont}[2]{\ensuremath{M_{#1, #2}}}
\newcommand{\Id}{\mathcal{O}}

% other shortcuts 
\newcommand\todo[1]{(\textcolor{red}{{\bf todo}}: #1)}
\def\td{(\textcolor{red}{{\bf todo}})}
\newcommand\done[1]{- \textcolor{red}{{\bf done}} #1}
\def\done{(\textcolor{red}{{\bf done}})}
\DeclarePairedDelimiter{\ceil}{\lceil}{\rceil}
\DeclarePairedDelimiter{\floor}{\lfloor}{\rfloor}
\DeclareMathOperator{\cswap}{cswap}
\DeclareMathOperator{\End}{End}
\DeclareMathOperator{\chr}{char}
\DeclareMathOperator{\bd}{bd}
\DeclareMathOperator{\crit}{crit}

% HURL
\usepackage{hyperref}
\hypersetup{
    colorlinks=true,
    linkcolor=red,
    filecolor=red,      
    urlcolor=red,
    citecolor=red
}
\urlstyle{same}



\renewcommand{\thefootnote}{\fnsymbol{footnote}}

% Algorithms
\usepackage[vlined,ruled,linesnumbered,titlenotnumbered]{algorithm2e}
%\usepackage{algorithm2e}
\newcommand{\Input}{\item[\textsc{Input:}]}
\newcommand{\Output}{\item[\textsc{Output:}]}
\newcommand{\Remark}{\item[\textsc{Remark:}]}
\newcommand{\pushline}{\Indp}% Indent
\newcommand{\popline}{\Indm\dosemic}% Undent
\let\oldnl\nl% Store \nl in \oldnl
\newcommand{\nonl}{\renewcommand{\nl}{\let\nl\oldnl}}% Remove line number for one line
\makeatother

% Misc Packages
\usepackage{mathtools}
\usepackage{hyperref}
\usepackage{xargs}
\usepackage[symbol]{footmisc}

% Tikz
\usepackage{tikz}
\usetikzlibrary{arrows.meta}



% 
\def\td{(\textcolor{red}{{\bf todo}}) }
\def\why{(\textcolor{red}{{\bf why?}}) }








\title{Removing the Noether position requirement}
\author{}
\date{}



\begin{document}

\maketitle

For $i$ in
$\{1,\dots,n\},$ consider the projections
\begin{align*}
    \vp_i: \C^n  &\rightarrow \C^i \\
    (x_1,\hdots,x_n) &\mapsto  (x_1,\hdots,x_i)    
\end{align*}
\begin{align*}
    \psi_i: \C^n  &\rightarrow \C^{n-1} \\
    (x_1,\hdots,x_n) &\mapsto  (x_1,\hdots,x_{i-1},x_{i+1},\hdots,x_n)    
\end{align*}
\begin{align*}
    \pi_i: \C^n  &\rightarrow \C \\
    (x_1,\hdots,x_n) &\mapsto x_i.
\end{align*}




\begin{proposition}\label{prop:main}
    Let $V \subset \C^n$ be a smooth affine variety and let $C$ be a connected component of $V \cap \R^n$ with dimension $d$. If, for all $i \in \{1,\hdots,n\}, C \cap \crit(\pi_i,V) = \emptyset$  and  $\pi_i(C) \not = \R$, then there exists distinct $i,j \in\{1,\hdots,n\}$ with
    \[
C \cap \{ x_i= x_j\} \not = \emptyset \textrm{ or }
C \cap \{ x_i= -x_j\} \not = \emptyset. 
    \]
\end{proposition}




\section*{First argument for Proposition \ref{prop:main}}









\begin{proof}
First assume that $\dim(C)=1$ and assume $n=2$. Since $\pi_1(C) \not = \R$ and $C \not = \emptyset$, there exists 
\[
a \in \bd(\pi_1(C)) = \overline{\pi_1(C)} - \pi_1^o(C).
\]
And since $C \cap \crit(\pi_1,C) = \emptyset,$ we know that $a \not \in \pi_1(C).$ Now, it follows that there exists a sequence $(\alpha_l)_{l \in \N} \subset C$ with $\pi_1(\alpha_l) \rightarrow a$ as $l \rightarrow \infty$ and $|| \alpha_l|| \rightarrow \pm\infty$ as $l \rightarrow \infty.$ It must therefore be that 
\[
\pi_2(\alpha_l) - \pi_1(\alpha_l) \rightarrow \pm\infty ~(assume +\infty~ WLOG).
\]
Now, similarly, there exists 
\[
b \in \bd(\pi_2(C)) = \overline{\pi_2(C)} - \pi_2^o(C), b \not \in \pi_2(C).
\]
Let $(\beta_k)_{k \in \N} \subset C$ with $\pi_2(\beta_k) \rightarrow b$ and $|| \beta_k|| \rightarrow \infty.$ And clearly we have that 
\[
\pi_1(\beta_k) - \pi_2(\beta_k) \rightarrow \infty
\]
so that $\pi_1(\beta_k) \rightarrow \infty.~\hdots$
%
%
%
\par 
Now assume $n>2.$ Let 
\[
N = \big\{i \in \{1,\hdots,n\} ~|~ \psi_i(C) \textrm{ is closed}\big\}. 
\]
\begin{enumerate}
    \item $N = \emptyset:$ 
    \par 
    Since $N$ is empty, $\vp_{n-1}(C)$ is not closed. It follows that $\dim(\vp_{n-1}(C))=1$ because, otherwise, since $C$ is non-empty  and since dimension can only decrease after projection, $\dim(\vp_{n-1}(C))=0$ implying that $\vp_{n-1}(C)$ is closed, a contradiction \lightning. Therefore, it follows that $\alpha \in \R^{n-1}$ exists with 
    \[
    \alpha \in \bd(\vp_{n-1}(C)) = \overline{\vp_{n-1}(C)} - \vp_{n-1}^o(C),~ \alpha \not \in \vp_{n-1}(C).
    \]
    In other words, $\alpha$ lies in the set of non-properness of $\vp_{n-1}|_C$. And therefore, a sequence $(\zeta_l)_{l \in \N} \subset C$ exists with $\vp_{n-1}(\zeta_l) \rightarrow \alpha$ and $\pi_n(\zeta_l) \rightarrow \infty.$ And, since, by assumption, 
\[
    C \cap \crit(\pi_n,V) = \emptyset \textrm{ and  } \pi_n(C) \not = \R,
\]
the projection $\pi_n(C)$ cannot be a bounded and open interval:
\[
\pi_n(C) = (e_n,+\infty) \textrm{ or } \pi_n(C) = (-\infty,e_n). 
\]
Now let $(\beta_l)_{l \in \N} \subset C$ with $\pi_n(\beta_l) \rightarrow e_n$. Then, since $\pi_n(C)$ is an open interval with boundary point $e_n$, there exists $i \in \{1,\hdots,n-1\}$ with $\pi_i(\beta_l) \rightarrow \infty$ and, again since \[
    C \cap \crit(\pi_i,V) = \emptyset \textrm{ and  } \pi_i(C) \not = \R
\]
and $\pi_i(\beta_l) \rightarrow \infty$,
\[
\pi_i(C) = (e_i,+\infty) \textrm{ or } \pi_i(C) = (-\infty,e_i). 
\]
Now pick a sequence of points in $C$ who's $i$-th projection converges to $e_i:$ 
\[
(\gamma_l)_{l \in \N} \subset C,~ \pi_i(\gamma_l) \rightarrow e_i.
\]
It then follows that $\vp_{n-1}(\gamma_l) \rightarrow \alpha$. Indeed, since $\pi_i(\gamma_l) \rightarrow e_i$, it must be that 
\[
e_i \in \bd(\vp_{n-1}(C)) = \overline{\vp_{n-1}(C)} - \vp_{n-1}^o(C).
\]
To $l$ associate 
\[
l \mapsto \vp_{n-1}(C) \cap B(\alpha,1/l).
\]
For $l$ large enough, the intersection is not empty and
%; when you find one point there, lift it and define $(\gamma_l)$ this way. 
\[
e_i =\pi_i\circ \vp_{n-1}(\alpha).
\]
We then have that $C$ will either intersect $x_n-x_i$ or $C$ will intersect $x_n+x_i$. By continuity, because $C$ is connected $x_i - x_n$ or $x_i+x_n$ has a sine change over $C$ (positive at one point, negative at another, and therefore 0 at someplace). 
%
\item $N \not = \emptyset:$
\par 
Mimic the same reasoning over $\psi_i(C)$. Replace $C$ with $\psi_i(C)$ and WLOG assume that $i$ is $n$ so we work in dimension $n-1$ with a closed semi-algebraic curve in $\R^{n-1}$. By an injection argument, we end up with the $n=2$ case (an inductive argument can be made with with having $n=2$ as the base case - \td).  
\end{enumerate}


\end{proof}



\section*{Second argument for Proposition \ref{prop:main}}
\begin{lemma}
    There exists $i \in \{1,\hdots,n\}$ with $\pi_i(C)$ not bounded. WLOG assume $i=1.$
\end{lemma}
\begin{proof}
    \td
\end{proof}
\noindent 
Consider the following two cases. 
\begin{enumerate}
    \item $\vp_2(C)$ is closed:
\newline 
Then as $x_1 \rightarrow e_2, x_2 \rightarrow \infty$, and as  $x_2 \rightarrow e_1, x_1 \rightarrow \infty$. Therefore, we have a sign change of $x_1-x_2$ over the connected component. 
    \item $\vp_2(C)$ is not closed:
\newline 
When approaching $\alpha$, some coordinate $x_i \rightarrow \infty.$ WLOG assume that $i=3.$ It must also be that when approaching $x_3$, $x_1 \rightarrow \infty$ because otherwise there is a boundary point and $\pi_1(C)$ is bounded, contradicting our assumption \lightning. Therefore, $x_1-x_3, x_1+x_2$ or $x_1+x_3$ (depending on whether or not we go to $\pm \infty$) will have a sign change over the connected component.
\end{enumerate}
\paragraph*{Comments:}
\begin{itemize}
    \item[--] Insight to generalize to the $d$-dimensional case? 
    \item[--] It (the proof) tells us that we can read everything on some projection of dimension $d+1$.
    \item[--] Gives a new ingredient: we can get the information by looking at the projections on spaces of dimension $d+1.$ 
    \item[--] Our claim (in dimension 1) is basically that there is a 1 dimensional coordinate space such that the projection on the 1 dimensional coordinate space is unbounded.... could we prove (the following lemma)?
\end{itemize}



\begin{lemma}
    Where $d = \dim (C),$ there exists $\{i_1,\hdots,i_d\} \subset \{1,\hdots,n\}$ with $\pi_j(C)$ unbounded (infinite?) for all $j \in \{i_1,\hdots,i_d\}.$
\end{lemma}
\begin{proof}
    There exists $d$ generic hyperplnes $H_1,\hdots,H_d$ with $C \cap H_1 \cap \cdots \cap H_d$ finite, and the intersection of $C$ with any fewer of these hyperplanes is infinite. For each $k \in \{1,\hdots,d\}, C \cap H_k$ therefore also gives a distinct coordinate $x_{i_k}$ \color{red} (true?) \color{black} for which the projection $\pi_{i_k}(C)$ is infinite. Hence, we have $\{i_1,\hdots,i_d\} \subset \{1,\hdots,n\}$ with $\pi_j(C)$ infinite for all $j \in \{i_1,\hdots,i_d\}.$
\end{proof}









\end{document}
