\documentclass[11pt]{article}
% packages
\usepackage[left=1.87cm,right=1.87cm,top=1.87cm,bottom=1.87cm]{geometry}
\usepackage[utf8]{inputenc}
\usepackage{mathrsfs}
\usepackage{multirow}
\usepackage{amsfonts}
\usepackage{amssymb}
\usepackage{amsmath}
\usepackage{float}
\usepackage{mathtools}
\usepackage{amsthm} 
\usepackage{comment}
\usepackage{xcolor}
\usepackage{graphicx}
\usepackage{yhmath}
\usepackage{mathdots}
\usepackage{wasysym}
\usepackage{xargs}
\usepackage[symbol]{footmisc}

% HURL
\usepackage{hyperref}
\hypersetup{
    colorlinks=true,
    linkcolor=red,
    filecolor=red,      
    urlcolor=red,
    citecolor=red
}
\urlstyle{same}

% Theorem environments
\theoremstyle{definition}
\newtheorem{theorem}{Theorem}
\newtheorem{claim}{Claim}
\newtheorem{proposition}{Proposition}
\newtheorem{lemma}{Lemma}
\newtheorem{remark}{Remark}
\newtheorem{definition}{Definition}

% Math commands
\newcommand{\F}{\mathbb{F}}
\newcommand{\N}{\mathbb{N}}
\newcommand{\C}{\mathbb{C}}
\newcommand{\R}{\mathbb{R}}
\renewcommand{\P}{\mathbb{P}}
\newcommand{\Fp}{\mathbb{F}_p}
\newcommand{\Fpp}{\mathbb{F}_{p^2}}
\newcommand{\vp}{\varphi}
\newcommand{\ra}{\rightarrow}
\newcommand{\vph}{\hat{\varphi}}
\newcommand{\Z}{\mathbb{Z}}

% Keywords command
\providecommand{\keywords}[1]
{
  \smallskip\noindent\small	
  \textbf{\textbf{Keywords---}} #1
}
\newcommand{\LC}[2]{\ensuremath{L_{[#1 : #2]}}}
\newcommand{\Mont}[2]{\ensuremath{M_{#1, #2}}}
\newcommand{\Id}{\mathcal{O}}

% other shortcuts 
\newcommand\todo[1]{(\textcolor{red}{{\bf todo}}: #1)}
\def\td{(\textcolor{red}{{\bf todo}})}
\newcommand\done[1]{- \textcolor{red}{{\bf done}} #1}
\def\done{(\textcolor{red}{{\bf done}})}
\def\td{(\textcolor{red}{{\bf todo}}) }
\def\why{(\textcolor{red}{{\bf why?}}) }
\renewcommand{\thefootnote}{\fnsymbol{footnote}}
\DeclarePairedDelimiter{\ceil}{\lceil}{\rceil}
\DeclarePairedDelimiter{\floor}{\lfloor}{\rfloor}
\DeclareMathOperator{\cswap}{cswap}
\DeclareMathOperator{\End}{End}
\DeclareMathOperator{\chr}{char}
\DeclareMathOperator{\bd}{bd}
\DeclareMathOperator{\crit}{crit}


% Algorithms
\usepackage[vlined,ruled,linesnumbered,titlenotnumbered]{algorithm2e}
%\usepackage{algorithm2e}
\newcommand{\Input}{\item[\textsc{Input:}]}
\newcommand{\Output}{\item[\textsc{Output:}]}
\newcommand{\Remark}{\item[\textsc{Remark:}]}
\newcommand{\pushline}{\Indp}% Indent
\newcommand{\popline}{\Indm\dosemic}% Undent
\let\oldnl\nl% Store \nl in \oldnl
\newcommand{\nonl}{\renewcommand{\nl}{\let\nl\oldnl}}% Remove line number for one line
\makeatother


% Tikz
\usepackage{tikz}
\usetikzlibrary{arrows.meta}




%\title{A las vegas algorithm algorithm for computing one point in each connected component of a smooth real algebraic set}
%\author{Jesse Elliott\thanks{David R. Cheriton School of Computer Science, University of Waterloo, On, Canada}, Mohab Safey El Din\thanks{Sorbonne Universit\'e, CNRS, INRIA, Laboratoire d’Informatique de Paris 6, PolSys, Paris, France}, \'Eric Schost \thanks{David R. Cheriton School of Computer Science, University of Waterloo, On, Canada} 
%}


\date{}

\begin{document}

%\maketitle




%\begin{abstract}
%This is an abstract \newline
%\keywords{Real algebraic geometry, polynomial system solving, algorithms}
%\end{abstract}


\newpage 


%\section{}


For $\{i_1,\hdots,i_d\} \subset \{1,\dots,n\},$ consider the projections
\begin{align*}
    \pi_{i_1,\hdots,i_d}: \C^n  &\rightarrow \C^d \\
    (x_1,\hdots,x_n) &\mapsto  (x_{i_1},\hdots,x_{i_d})    
\end{align*}
%
\begin{proposition}\label{prop:main}
Let $V \subset \C^n$ be a smooth variety and let $C$ be a connected component of the real trance $V \cap \R^n$ with dimension $d$. If, for all $\{i_1,\hdots,i_d\} \subset \{1,\dots,n\}$, 
\begin{enumerate}
    \item $C \cap \crit(\pi_{i_1,\hdots,i_d},V) = \emptyset$;
    \item $\pi_{i_1,\hdots,i_d}(C) \not = \R^d,$
\end{enumerate}
then there exists distinct $i,j \in\{1,\hdots,n\}$ with
    \[
C \cap \{ x_i= x_j\} \not = \emptyset \textrm{ or }
C \cap \{ x_i= -x_j\} \not = \emptyset. 
    \]
\end{proposition}
%
\begin{proof}




\begin{lemma}
    There exists $i \in \{1,\hdots,n\}$ with $\pi_i(C)$ not bounded. 
\end{lemma}
\begin{proof}
    Assume for contradiction that $\pi_i(C)$ is bounded for all $i \in \{1,\hdots,n\}$, let  
    \[
\pi_i(C) = (e_i,e_i').
    \]
Notice also that $C$ is closed, and therefore $C$ is compact which implies that $\pi_i(C) = (e_i,e_i')$ are compact which is a contradiction \lightning.
\end{proof}
\noindent
WLOG assume $i=1$ and $\pi_1(C) = (e_1,+\infty).$ We obtain the following parameterization of $C$: 
%
\[
\big\{\big(t_1,\hdots,t_d,g_{n-d}(t_1,\hdots,t_d),\hdots,g_{n}(t_1,\hdots,t_d)\big)~|~ t_1,\hdots,t_d \in \color{red}~?\color{black}\big\} \subset C
\]
%
Assuming that (1) from Proposition \ref{prop:main} implies that 
\[
g_{n-d}(t_1,\hdots,t_d),\hdots,g_{n}(t_1,\hdots,t_d)
\]
are monotonic, use (2) from Proposition \ref{prop:main} to show that, for some $i\not =j \in \{1,\hdots,n\}$, $C$ meets 
\[
x_i \pm x_j = 0.
\]



    
\end{proof}























\end{document}
